\section{Simulationsumgebungen für RL}

Anders als im klassischen Bereich des maschinellen Lernens wie dem überwachten- und unüberwachten Lernen, werden beim verstärkenden Lernen viele der Testdatensätze nicht aus der echten Welt akquiriert. \footcite[Vgl.][S. 1]{Zhang.2018}
Um entsprechend realistische Daten für das Training bereitzustellen, werden Simulationsumgebungen in Abhängigkeit von ihrer RL Anwendung ausgewählt. \footcite[Vgl.][S. 7]{Korber.2021}
Dennoch bleibt nahezu immer eine gewisse Diskrepanz zwischen der Dynamik in der Simulation und der Dynamik in der echten Welt. \footcite[Vgl.][S. 1]{Bharadhwaj.2019}
Möglichkeiten, diese Diskrepanz zu minimieren, sind zum einen das Fehlverhalten von Sensoren einzubinden oder ein reales Signal mit der virtuellen Umgebung zu verknüpfen. \footcite[Vgl.][S. 1]{Zhang.2018}
Dennoch lässt sich kaum garantieren, dass erlernte Strategien der Agenten sich auf nur leicht veränderte Umgebungen übertragen lassen. \footcite[Vgl.][S. 1]{Bharadhwaj.2019}
Anders als in der Simulation von Flüssigkeiten und deren Dynamik, bedarf RL eine reaktive Umgebung, dessen Verhältnis der Simulationszeit zur echten Zeit mindestens eins oder darüber liegt. \footcite[Vgl.][S. 3]{Korber.2021}
RL kann von einem erhöhten Echtzeitfaktor profitieren, trotz der damit einhergehenden verringerten Präzision. \footcite[Vgl.][S. 3]{Korber.2021}

\subsection{Definitionen von Simulationsumgebungen}
Ausgehend von der Literaturrecherche zeigt sich, dass in der Forschungsliteratur die allgemeine Definition von Simulationen kaum aufgegriffen wird.
Eine mögliche Definition nach \cite[]{Maria.1997} wird wie folgt dargelegt:

\textit{Eine Simulation eines existierenden Systems stellt die Anwendung eines Modells dar, welches konfigurierbar zu experimentellen Zwecken das eigentliche System vertritt, um wirtschaftliche oder systematische Herausforderungen des existierenden Systems zu umgehen.
Das Model wird in diesem Kontext definiert als Repräsentation des Aufbaus und der Verhaltensweise des existierenden Systems.}

Innerhalb bestimmter Anwendungsgebiete, wie der Medizin und der Pflege, werden zusätzlich virtuelle Simulationen wie nachstehend definiert.

\textit{Unter virtuellen Simulationen versteht man eine digitale Lernumgebung, welche durch teilweiser Immersion eine wahrnehmbare Erfahrung bereitstellt.\footcite[Vgl.][S. 1]{Foronda.2021}}

\subsection{Entwicklung von Simulationsumgebungen für RL Anwendungen}
Im weiteren Teil dieses Kapitels wird aufbauend auf den zuvor angeführten Definitionen, die Entwicklung von Simulationen betrachtet.
Allgemein lässt sich dieser Entwicklungsprozess in die folgenden Teilschritte gliedern:\footcite[Vgl.][S. 8f.]{Maria.1997}
\begin{enumerate}
    \item Identifikation der Herausforderungen im existierenden System und Ableitung von Anforderungen für die Simulation.
    \item Zielgruppe, Funktionsrahmen und quantitative Bewertungskriterien der Simulation definieren.
    \item Analyse des zu simulierenden Systemverhaltens durch Sammeln und Verarbeiten von realen Daten des existierenden Systems.
    \item Entwicklung einer schematischen Darstellung des Modells und dessen Überführung in nutzbare Software.
    \item Validierung des Modells durch bspw. den Vergleich mit dem existierenden System.
    \item Dokumentierung des Modells, dessen Variablen, Metriken und getroffene Annahmen.
\end{enumerate}

Die Entwicklung von Simulationen wurde in der Forschungsliteratur besonders durch den Fortschritt im Bereich des verstärkenden Lernens vorangetrieben, da der Vergleich von RL-Algorithmen zuverlässige Benchmarks in Form von Simulationsumgebungen benötigt.\footcite[Vgl.][S. 1]{Brockman.2016}
Aus dieser Motivation wurde 2016 durch die \textit{OpenAI} der \textit{OpenAI Gym} Werkzeugkasten entwickelt, welcher eine Sammlung an Benchmarksimulationen mit einer einheitlichen Schnittstelle für RL Algorithmen enthält. \footcite[Vgl.][S. 1]{Brockman.2016}
Seither wurde diese definierte Schnittstelle vielfach verwendet, um RL Umgebungen mit dem Ziel zu entwickeln, diese zu publizieren und dessen Wiederverwendung zu ermöglichen. \footcite[Vgl.][S. 4]{Schuderer.2021}
Die Entwicklung dieses Softwareframeworks wurde nach der Version 0.26.0 im Jahr 2022 durch ein neues Team unter dem Namen Gymnasium weitergeführt. \footcite[Vgl.][]{GitHub.442023}
Die Schnittstelle ist definiert als Python Klasse \textit{gym.Env}, von welcher weitere Klassen erben und die vorgeschriebenen Funktionen zum Zeitschritt und zum Zurücksetzen der Simulation implementieren. \footcite[Vgl.][S. 4]{Schuderer.2021}
Der Werkzeugkasten von OpenAI fokussiert sich auf einen „Episoden ähnlichen Rahmen“, in welchem der Agent durch zunächst zufälliges Auswählen von Interaktionen lernt. \footcite[Vgl.][S. 1]{Brockman.2016}
Weitere Entwicklungsentscheidungen des OpenAI Gym Werkzeugkastens umfassen z. B. die bewusst fehlende Schnittstelle des Agenten, die strikte Versionierung der Umgebung oder die standardmäßige Simulationsüberwachung. \footcite[Vgl.][S. 2f.]{Brockman.2016} % Welche Schnittstelle des Agenten ist hier gemeint? Oder die Implementierung des Agenten?

Werden Lernumgebungen nach der Gym Schnittstelle oder nach eigener Definition für RL Anwendungen eingesetzt, kann sich deren Gestaltung unterschiedlich auf die Leistung der Anwendung auswirken. \footcite[Vgl.][S. 1]{Reda.2020}
Eine enge initiale Wahrscheinlichkeitsverteilung des Umgebungszustandes kann die Lerneffizienz erhöhen, wohingegen eine weite Wahrscheinlichkeitsverteilung positiv die Robustheit der erlernten Strategie beeinflusst. \footcite[Vgl.][S. 3]{Reda.2020}
Die Robustheit kann zusätzlich durch die Einbindung von Fehlverhalten in der Wahrnehmung der Umgebung beeinflusst werden, da auch in realen Szenarien ein Risiko für Fehlverhalten besteht. \footcite[Vgl.][S. 2]{YanDuan.2016} 
Im Bereich der Robotik bzw. in der Simulation von Bewegungen, kann auch durch die Gestaltung des Aktionsraumes, basierend auf elektrischer Regelungstechnik mittels PID-Regler, anstatt basierend auf Drehmomenten ein effizienterer Lernprozess stattfinden.\footcite[Vgl.][S. 7]{Reda.2020}

Neben den beschriebenen Eigenschaften von Umgebungen für verstärkendes Lernen unterliegen auch die verwendeten Simulationen bestimmten Merkmalen, welche in der Entwicklung zu berücksichtigen sind.
Laut einer Umfrage nach \cite[]{Ivaldi.2272014} sind diese wichtigsten Eigenschaften die Stabilität, Geschwindigkeit, Präzision, Genauigkeit, Bedienbarkeit und der Ressourcenverbrauch.
Die Entwicklung des Modells, welches das existierende System ersetzt, sollte sich demnach möglichst positiv auf die beschriebenen Eigenschaften auswirken.
Neben den beschriebenen leistungsbezogenen Merkmalen, sind die folgenden weiteren Kriterien mitunter die wichtigsten zur Auswahl einer Simulation:

\begin{table}[H]
    \centering
    \begin{tabular}{|l|l|}
    \hline
    Rank & Most important criteria                     \\ \hline
    1    & Simulation very close to reality            \\
    2    & Open-source                                 \\
    3    & Same code for both real and simulated robot \\
    4    & Light and fast                              \\
    5    & Customization                               \\
    6    & No interpenetration between bodies          \\ \hline
    \end{tabular}
    \caption{wichtigste Kriterien zur Auswahl von Simulatoren \footcite[][S. 4]{Ivaldi.2272014}}
    \label{tab:important-criteria}
\end{table}
\footnotetext{Enthalten in: \cite[][s. 4]{Ivaldi.2272014}}

Aus Tabelle 2 lässt sich entnehmen, dass besonders die Nähe zur Realität ein wichtiges Auswahlkriterium ist.
Im Kontext von verstärkendem Lernen im Robotik Bereich ist ein wichtiger Baustein die Physik-Engine zur Modellierung von Dynamiken. \footcite[Vgl.][S. 2]{Ayala.2020}

\subsection{Aktuelle Physik-Engines und Simulationsanwendungen}
Innerhalb dieses Abschnittes wird aufgrund der Bedeutung der Physik-Engine für den Grad der Simulationsrealität, eine Auswahl der aktuellen Physik-Engines und deren Simulationsanwendung betrachtet.

\textit{Gazebo}

Gazebo ist eine durch die Open Source Robotics Foundation entwickelte Simulationsanwendung, welche mehrere Physik-Engines unterstützt. \footcite[Vgl.][S. 7]{Ivaldi.2272014}
Mittels Gazebo lassen sich Interaktionen zwischen Robotern in Innen- und Außenbereichen unter realistischer Sensorik simulieren. \footcite[Vgl.][S. 4]{Ayala.2020}
Die unterstützten Physik-Engines umfassen Bullet, Dynamic Animation and Robotics Toolkit (DART), Open Dynamics Engine (ODE) and Simbody. \footcite[Vgl.][S. 3]{Korber.2021}

\textit{MuJoCo}

MuJoCo stellt eine Physik-Engine für modellbasierte Steuerung dar, dessen Objekte durch C++ oder XML definiert und Gelenkzustände im Koordinatensystem beschriebenen werden. \footcite[Vgl.][S. 1]{Todorov.2012} 
Diese beschriebene Eigenschaft lässt sich auch aus dem Namen als Abkürzung für \textbf{Mu}lti-\textbf{Jo}int dynamics with \textbf{Co}ntact ableiten. \footcite[Vgl.][S. 2]{Todorov.2012}
Die MuJoCo Anwendung ist lizensiert, was den Besitz einer Lizenz für die Installation oder die Virtualisierung innerhalb eines Containers voraussetzt. \footcite[Vgl.][S. 3]{Korber.2021}

\textit{PyBullet}

PyBullet basiert als Simulationssoftware auf der Bullet-Engine und fokussiert sich funktional auf die Anwendung von RL im Robotikbereich. \footcite[Vgl.][S. 3]{Korber.2021}
Die Bullet-Engine ist hingegen eine offene Softwarebibliothek, welche neben verstärkenden Lernen auch bei Computeranimationen angewendet werden kann. \footcite[Vgl.][S. 7]{Ivaldi.2272014}
Bie Handhabung von PyBullet profitiert von der ausgiebigen Dokumentation, der großen Entwicklergemeinschaft und der Unterstützung von verschiedenen Dateiformaten wie SDA, URDF und MJCF zur Einbindung von Objekten. \footcite[Vgl.][S. 6]{Korber.2021}

%\textbf{Hier noch mehr SimEnvs mit aufnehmen ?}