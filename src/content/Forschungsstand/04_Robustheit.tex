\section{Robustheit und Stabilität von Strategien des verstärkenden Lernens}

Anders als innerhalb von Simulationen lassen sich in der echten Welt häufig Unsicherheiten, Störeinflüsse und grundlegende Veränderungen der Umgebung wahrnehmen, für welche die Methoden des RL standardmäßig nicht robust genug sind. \footcite[Vgl.][S. 1]{Moos.2022}
Im nachfolgenden Kapitel wird daher genauer dargestellt, was unter der Robustheit von Algorithmen des verstärkenden Lernens verstanden wird, was Kenngrößen sind und wie man diese misst.

\subsection{Definitionen von Robustheit und Stabilität}

In der aktuellen Forschungsliteratur findet sich nur eine geringe Gemeinsamkeit innerhalb der unterschiedlichen Definitionen von Stabilität und Robustheit. \footcite[Vgl.][S. 5]{Pullum.2022}
Die Definition der Robustheit im Kontext von verstärkendem Lernen wird verschieden interpretiert, wie z.B. als Robustheit gegen Störeinflüsse, Beeinflussung der Belohnung, oder Umgebungsunterschiede. \footcite[Vgl.][S. 2]{Liu.2023}
\cite[]{Pullum.2022} definiert Stabilität und Robustheit im Kontext der Literaturanalyse wie folgt:

\textit{Stabilität ist eine Eigenschaft des lernenden Algorithmus, die sich auf dessen Leistungsvarianz bezieht und bei geringer Varianz auf ein stabiles Modell hinweist.} \footcite[Vgl.][S. 5]{Pullum.2022}

\textit{Robustheit im Kontext von Software, referenziert eine Eigenschaft eines System, welches nicht nur ausschließlich unter normalen, sondern auch unter außergewöhnlichen Bedingungen, welche die Annahmen des Entwicklers übersteigen, gut funktioniert.} \footcite[Vgl.][S. 5]{Pullum.2022}

\cite[]{Moos.2022} beschreibt die Robustheit in seiner Literaturanalyse hingegen als Fähigkeit mit Variationen und Unsicherheiten in der Umgebung umgehen zu können, wobei Unsicherheiten häufig variierende physische Parameter darstellen. \footcite[Vgl.][S. 1]{Moos.2022}

\subsection{Metriken der Robustheit}

\subsection{Messung der Robustheit}

