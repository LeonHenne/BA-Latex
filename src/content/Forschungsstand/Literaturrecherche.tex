\section{Aufbau der Literaturrecherche}

In Anlehnung der Literaturrecherche nach \cite[]{10.5555/2017160.2017162} wurden alle voraussichtlich benötigten Konzepte für die Durchführung der beschriebenen Forschungsmethodik in Tabelle 1 festgehalten.
Alle angeführten Konzepte wurden mittels verschiedener Suchbegriffe in Suchmaschinen, Datenbanken und Bibliotheken wie \cite[]{.2282023}, \cite[]{.2282023b} oder die digitale Bibliothek der Association for Computing Machinery (ACM) \cite[]{ACMDigitalLibrary.2282023}  recherchiert.
In der daraus gefunden Literatur wurden zitierte Werke ebenfalls nach den beschriebenen Konzepten durchsucht und insgesamt jede Literaturquelle in Tabelle 1 den in ihnen enthaltenen Konzepten zugeordnet.  
% Please add the following required packages to your document preamble:
% \usepackage{longtable}
% Note: It may be necessary to compile the document several times to get a multi-page table to line up properly
\begin{longtable}{|l|llllllll|}
    \hline
    \multicolumn{1}{|c|}{Artikel} & \multicolumn{8}{c|}{Konzepte}                                                                                                                                                                \\ \hline
    \endhead
                                  & \multicolumn{1}{l|}{RL} & \multicolumn{1}{l|}{MARL} & \multicolumn{1}{l|}{ES} & \multicolumn{1}{l|}{DS} & \multicolumn{1}{l|}{KS} & \multicolumn{1}{l|}{DR} & \multicolumn{1}{l|}{RRLP} & LE \\ \hline
                                  & \multicolumn{1}{l|}{}   & \multicolumn{1}{l|}{}     & \multicolumn{1}{l|}{}   & \multicolumn{1}{l|}{}   & \multicolumn{1}{l|}{}   & \multicolumn{1}{l|}{}   & \multicolumn{1}{l|}{}     &    \\ \hline
                                  & \multicolumn{1}{l|}{}   & \multicolumn{1}{l|}{}     & \multicolumn{1}{l|}{}   & \multicolumn{1}{l|}{}   & \multicolumn{1}{l|}{}   & \multicolumn{1}{l|}{}   & \multicolumn{1}{l|}{}     &    \\ \hline
                                  & \multicolumn{1}{l|}{}   & \multicolumn{1}{l|}{}     & \multicolumn{1}{l|}{}   & \multicolumn{1}{l|}{}   & \multicolumn{1}{l|}{}   & \multicolumn{1}{l|}{}   & \multicolumn{1}{l|}{}     &    \\ \hline
                                  & \multicolumn{1}{l|}{}   & \multicolumn{1}{l|}{}     & \multicolumn{1}{l|}{}   & \multicolumn{1}{l|}{}   & \multicolumn{1}{l|}{}   & \multicolumn{1}{l|}{}   & \multicolumn{1}{l|}{}     &    \\ \hline
                                  & \multicolumn{1}{l|}{}   & \multicolumn{1}{l|}{}     & \multicolumn{1}{l|}{}   & \multicolumn{1}{l|}{}   & \multicolumn{1}{l|}{}   & \multicolumn{1}{l|}{}   & \multicolumn{1}{l|}{}     &    \\ \hline
    \caption{Konzept Matrix für Artikel zur Simulation und zur Robustheit RL Algorithmen nach \cite[]{10.5555/2017160.2017162}.
    Legende: RL (Reinforcement Learning), MARL (Multi-Agent Reinforcement Learning), ES (Entwicklung von Simulationsumgebungen), DS (Dronensimulation), KS (kompetitive Simulationsumgebungen), DR (Domain Randomization), RRLP (Robustheit von RL Policies), LE (Laborexperimente)}
    \label{tab:research-table}\\
\end{longtable}