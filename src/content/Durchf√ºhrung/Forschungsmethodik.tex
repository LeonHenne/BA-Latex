Anschließend an die Diskussion des aktuellen Stands der Forschung und der Praxis wird im Rahmen dieses Kapitels die Forschungsmethodik angewandt, um die folgende Forschungsfrage zu beantworten:

\textit{Inwiefern kann durch den Einsatz eines mittels RL trainierten Gegenspielers die Robustheit der gelernten Strategie verbessert werden?}

% Die in der Forschungsfrage formulierte These untersucht eine mögliche Verbesserung des Sim-to-Real Problems durch trainiertem Gegenspieler.
% Für diese Untersuchung wird ein Laborexperiment nach \cite[]{Recker.2021} durchgeführt, indem die Robustheit der erlernten Strategien unter der Verwendung eines deterministischen und eines trainierten Gegenspielers betrachtet wird. 
% Die quantitative Auswertung mittels des Laborexperiments kann hierbei besonders die Ursache-Wirkung Beziehung zwischen dem gewählten Trainingsszenarien und der erzielten Robustheit untersuchen.

In der ersten Sektion dieses Kapitels wird auf den Aufbau des Laborexperiments eingegangen.
Dabei werden grundlegende Trainingsdaten sowie zu untersuchende Hypothesen, ausgeübte Einflüsse und erfasste Metriken beschrieben.
Anschließend werden daraus in der nächsten Sektion resultierende Anforderungen für die Entwicklung der Simulation und der Testumgebung abgeleitet. 
Die dritte Sektion beinhaltet daraufhin die Umsetzung der beschriebenen Anforderungen und erläutert die endgültige Implementierung der Simulation, der Testdatenerhebung und deren Auswertung.

\section{Erläuterung der Forschungsmethodik}

Innerhalb des Laborexperiments soll die Beziehung zwischen dem Trainingsszenario als Ursache und der Leistungsrobustheit als Wirkung betrachtet werden.
Dabei sind zunächst Strategien durch verstärkendes Lernen unter unterschiedlichen Szenarien zu optimieren. 
Anschließend werden die trainierten Policies in einer veränderten Simulation als Testszenario ausgeübt, und währenddessen verschiedene Metriken der Strategieleistung betrachtet.
Ziel ist es, die Leistungsdiskrepanzen zwischen den Strategien des Trainings mit deterministischen und mit optimiertem Gegenspieler zu betrachten.
Kein Teil der umzusetzenden Forschungsmethodik ist die Anwendung der simulationsbasierten Strategien zur Steuerung von Quadrokoptern in der echten Welt. 
Weiterhin wird nicht die Ergebnisabhängigkeit zu Faktoren wie der Wahl des Simulationsframeworks, der Physik-Engine oder des Abstraktionsniveaus untersucht.
Die Auswahl jener Aspekte wird unter den Gesichtspunkten der Softwarearchitektur getroffen, dessen Anforderungen und Implementierung in nachfolgenden Sektionen behandelt wird.

\subsection{Beschreibung der Simulationsumgebung}

Beginnend mit dem Training der Policies durch verstärkendes Lernen, wird hierfür anders als bei überwachtem und unüberwachtem Lernen kein unmittelbar vorliegender Datensatz benötigt.
Anstelle dessen basiert das Training auf der vollständigen oder teilweisen Wahrnehmung einer Lernumgebung, welche den lernenden Agenten für ausgeführte Aktionen positiv oder negativ belohnt.
Wie auch in der Einleitung erwähnt wird im Kontext dieser Arbeit die Simulation von Quadrokoptern dafür verwendet, die Lernumgebung und damit die Trainingsdaten für die Algorithmen des RL zur Verfügung zu stellen.
Die Simulation von Quadrokoptern stellt ein hochdynamisches Anwendungsgebiet dar, bei dem von einer hohen Diskrepanz zwischen der echten und simulierten Welt ausgegangen werden kann. 
Die Simulationsumgebung stellt grundsätzlich ein Szenario dar, in welchem zwei verschiedene Drohnen kompetitiv gegeneinander agieren.
Das Ziel einer Drohne ist es zu einem festgelegten Punkt hinter der zweiten Drohne zu gelangen, währenddessen die zweite Drohne versucht die angreifende Drohne auf seinem Weg abzufangen. 
Die verteidigende Drohne verwendet dafür das Mittel einer bewussten Kollision.
Beide Drohnen werden unter der Einschränkung ihrer Nähe zum Zielpunkt zufällig in einem drei dimensionalem Raum initialisiert, welcher nur horizontal einseitig beschränkt wird.
Durch die einseitige Beschränkung des Flugraums wird der natürliche Untergrund dargestellt.
Zu Beginn soll die verteidigenden Drohne näher am Zielpunkt als die angreifende Drohne starten, sodass stets ein Abfangen möglich ist.
Ist das Ziel einer der beiden Drohnen erreicht, oder besteht Kontakt mit dem Untergrund, so wird die Simulation beendet.

\subsection{Erläuterung der Trainingsszenarien}

Durch die erläuterte Simulation werden für das Laborexperiment Strategien in vier unterschiedlichen Szenarien mit einem Algorithmus des verstärkendem Lernen optimiert.
%Zur Prüfung der Zuverlässigkeit werden je Trainingsszenario zwei optimale Policies durch zwei verschiedene RL-Algorithmen erlernt.
Jede der Strategien wird dabei in einer der nachfolgenden Simulationsszenarien optimiert, dessen Auswahl die unabhängige Variable des Laborexperiments darstellt.
\begin{enumerate}
    \item Das erste Szenario beinhaltet das Training der zu verteidigenden Drohne gegen eine angreifende Drohne, welche eine deterministische regelbasierte Strategie ausführt.
    \item Im zweiten Szenario wird eine deterministische Strategie für die anzugreifende Drohne mittels RL optimiert, während die verteidigende Drohne anhand ihrer zuvor optimierten Strategie agiert.
    \item Anschließend enthält das dritte Szenario das Training einer Verteidigungsstrategie im kompetitiven Spiel mit der anzugreifenden Drohne unter zuvor optimierter Policy.
    \item Zum Abschluss wird das erste Szenario mit regelbasiertem Gegenspieler unter DR, also unter der Randomisierung dynamischer Parameter, als viertes Szenario wiederholt. 
    zur Randomisierung der Trainingsdomäne wird ein konstanter Windeffekt simuliert, dessen Richtung und Stärke zu jeder Trainingsepisode variiert. 
\end{enumerate}

\subsection{Erläuterung des Testszenarien}

Nach den Trainingsphasen werden einzelne Policies in mehreren Episoden, eines vom Training abweichenden Testszenarios, ausgeführt und deren Leistungsverhalten gemessen.
Eine Episode entspricht dabei einer Simulation des Testszenarios bis zu ihrem erfolgreichen oder nicht erfolgreichem Ende durch Drohnen-, Ziel-, oder Bodenkontakt.
Zusätzlich zu den Trainingsszenarien ist neben den initialen Startpositionen der Drohnen auch der von der Angreiferdrohne zu erreichende Zielpunkt zufällig zu verändern.
Neben der zusätzlichen Randomisierung des Zielpunktes wird außerdem ein zum Trainingsszenario verbesserter regelbasierter Gegenspieler verwendet.

\subsection{Messung der Robustheit von RL Policies}

Die in den Testszenarien betrachteten abhängigen Variablen sind die erzielte kumulierte Belohnung, dessen Varianz sowie die Anzahl an unbeabsichtigten Abstürzen.
Die kumulierte Belohnung und dessen Varianz stellen wie bereits beschrieben in der Forschung wichtige Kenngrößen dar, um die Robustheit von RL Policies zu bestimmen.
Als Robustheit wird im Rahmen dieser Arbeit die Signifikanz und Stabilität einer Leistungsdiskrepanz von Modellen zwischen Trainings- und Testszenarien definiert.
Die Metrik der Anzahl von unbeabsichtigten Abstürzen spiegelt im behandelten Anwendungsfall zusätzlich das Fehlschlagen der trainierten Strategie wider.
Mit der Auswahl der Metriken werden die Strategieeigenschaften der maximalen Strategieleistung, der Leistungsabweichung und der Strategiesicherheit deutlich.
Aus der Abweichung zwischen dem Leistungsverhalten während des Trainings und während des Tests, soll so die Robustheit von RL Policies erkennbar und messbar gestaltet werden.
Der Fokus liegt dabei auf der Messung der Robustheit von den optimierten Strategien aus Szenario eins, drei und vier.
Aus dem Vergleich des Leistungsverhaltens zwischen Strategie eins und drei kann eine mögliche Verbesserung der Robustheit gegenüber unbekannten Szenarien abgeleitet werden.
Anschließend kann der erzielte Effekt mit der Abweichung zwischen eins und vier verglichen, und so aktuell verwendete Methoden zur Erhöhung der Robustheit einbezogen werden.

\subsection{Auswertung mittels statistischer Tests}

Durch den Vergleich einzelner Strategien mittels dieser Vorgehensweise werden im Laborexperiment die zu Beginn der Arbeit aufgestellten Hypothesen auf ihre Gültigkeit untersucht.
Die nachfolgenden bereits zu Beginn angeführten Hypothesen beinhalten wie im vorherigen Kapitel diskutiert, verschiedene Aspekte der Robustheit von RL Strategien, welche in späteren Abschnitten einzeln ausgewertet werden.
\begin{enumerate}
    % verschiedene Aspekte: max. Reward, Stabilität des Rewards, Fehlschlag von Strategien
    \item \textit{Die im Testszenario erzielte kumulierte Belohnung ist unter Verwendung der Policy aus dem Training mit RL basiertem Gegenspieler signifikant und zuverlässig höher als die Policy aus dem Training mit regelbasiertem Gegenspieler.}
    \item \textit{Die Varianz der im Testszenario erzielten Belohnung ist unter Verwendung der Policy aus dem Training mit RL basiertem Gegenspieler signifikant und zuverlässig geringer als die Policy aus dem Training mit regelbasiertem Gegenspieler.}
    \item \textit{Die im Testszenario erreichte Anzahl von unbeabsichtigten Abstürzen ist unter Verwendung der Policy aus dem Training mit RL basiertem Gegenspieler signifikant und zuverlässig geringer als die Policy aus dem Training mit regelbasiertem Gegenspieler.}
\end{enumerate}

Jede der drei oben genannten Behauptungen wird in zwei statistischen Tests ausgewertet.
Dabei wird zum einen ein Signifikanztest zur Ungleichheit der Verteilungen und zum anderen ein Signifikanztest zur positiveren Verteilung der Metrik durch RL basierten Gegenspieler durchgeführt.
Die Thesen zur ungleichen oder positiveren Verteilung der Metrik durch trainierten Gegenspieler werden in den Signifikanztests als H0 Hypothese eingesetzt.
Die entsprechenden Gegenhypothesen formulieren jeweils die gegensätzliche Annahme mit regelbasiertem Gegenspieler. 
Fasst man diesen Aufbau der statistischen Signifikanztests zusammen, können folgende Testhypothesen festgelegt werden. 

\textbf{Ungleichheit des ersten Messwerts}
\begin{itemize}
    \item \textbf{Hypothese H0:} \textit{Die erzielte kumulierte Belohnung im dritten Testszenario ist signifikant unterschiedlich zu der des ersten Testszenarios.}
    \item \textbf{Hypothese H1:} \textit{Die erzielte kumulierte Belohnung im dritten Testszenario ist signifikant gleich zu der des ersten Testszenarios.}
\end{itemize}

\textbf{Verbesserung des ersten Messwerts}
\begin{itemize}
    \item \textbf{Hypothese H0:} \textit{Die erzielte kumulierte Belohnung im dritten Testszenario ist signifikant höher zu der des ersten Testszenarios.}
    \item \textbf{Hypothese H1:} \textit{Die erzielte kumulierte Belohnung im dritten Testszenario ist signifikant geringer zu der des ersten Testszenarios.}
\end{itemize}

\textbf{Ungleichheit des zweiten Messwerts}
\begin{itemize}
    \item \textbf{Hypothese H0:} \textit{Die Varianz der erzielten Belohnung im dritten Testszenario ist signifikant unterschiedlich zu der des ersten Testszenarios.}
    \item \textbf{Hypothese H1:} \textit{Die Varianz der erzielten Belohnung im dritten Testszenario ist signifikant gleich zu der des ersten Testszenarios.}
\end{itemize}

\textbf{Verbesserung des zweiten Messwerts}
\begin{itemize}
    \item \textbf{Hypothese H0:} \textit{Die Varianz der erzielten Belohnung im dritten Testszenario ist signifikant geringer zu der des ersten Testszenarios.}
    \item \textbf{Hypothese H1:} \textit{Die Varianz der erzielten Belohnung im dritten Testszenario ist signifikant höher zu der des ersten Testszenarios.}
\end{itemize}

\textbf{Ungleichheit des dritten Messwerts}
\begin{itemize}
    \item \textbf{Hypothese H0:} \textit{Die erreichte Anzahl von unbeabsichtigten Abstürzen ist im dritten Testszenario ist signifikant unterschiedlich zu der des ersten Testszenarios.}
    \item \textbf{Hypothese H1:} \textit{Die erreichte Anzahl von unbeabsichtigten Abstürzen ist im dritten Testszenario ist signifikant gleich zu der des ersten Testszenarios.}
\end{itemize}

\textbf{Verbesserung des dritten Messwerts}
\begin{itemize}
    \item \textbf{Hypothese H0:} \textit{Die erreichte Anzahl von unbeabsichtigten Abstürzen ist im dritten Testszenario ist signifikant geringer zu der des ersten Testszenarios.}
    \item \textbf{Hypothese H1:} \textit{Die erreichte Anzahl von unbeabsichtigten Abstürzen ist im dritten Testszenario ist signifikant höher zu der des ersten Testszenarios.}
\end{itemize}

Die Auswahl des Signifikanztests wird anhand der Wahrscheinlichkeit einer vorliegenden Normalverteilung vorgenommen.
Kann nach einem Kolmogorov- oder Shapiro- Test zu einem Signifikanzniveau von 10\% eine Normalverteilung angenommen werden, lässt sich zur Überprüfung der Hypothesen ein T-Test einsetzten.
Ist keine Normalverteilung gegeben, wird ein Mann-Whitney U Test verwendet.
Liegt schlussendlich der P-Wert des T- oder Mann-Whitney U Signifikanztests zur Prüdung eines Unterschieds unter 10\%, so wird die H0 Hypothese angenommen und die Abweichung in den Robustheitsdaten als signifikant betrachtet.
Zeigen die anschließenden statistischen Signifikanztests, dass eine Metrik sich durch die Auswahl des Trainingsszenarios mit trainiertem Gegenspieler verbessert, wird die Verbesserung als signifikant wahrgenommen, und die Hypothese bestätigt.
Die Beantwortung der Forschungsfrage erfolgt abschließend mit der Betrachtung der angenommenen oder abgelehnten Forschungshypothesen.
So werden die verschiedenen Merkmale der RL Policies untersucht und es kann festgestellt werden, welche Effekte auf Robustheit erzielt worden sind.
