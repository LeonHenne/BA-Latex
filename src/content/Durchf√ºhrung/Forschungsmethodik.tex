Anschließend an die Diskussion des aktuellen Stands der Forschung und der Praxis wird im Rahmen dieses Kapitels die Forschungsmethodik durchgeführt, um folgende Forschungsfrage zu beantworten:

\textit{Kann durch den Einsatz eines mittels RL trainierten Gegenspielers die Robustheit der gelernten Policy verbessert werden?}

Dazu wird die in der Einleitung beschriebene Zielsetzung bearbeitet, ein Experiment zur Robustheit der erlernten Strategien unter deterministischem und trainiertem Gegenspieler durchzuführen.
In der ersten Sektion wird auf den Aufbau des Experiments eingegangen, indem beschrieben wird, welche Metriken gemessen werden und wie der Einfluss auf die Laborumgebung ausgeübt wird.
Anschließend werden daraus resultierende Anforderungen für die Entwicklung der Testumgebung abgeleitet und in der nachfolgenden Sektion dargelegt. 
Die dritte Sektion setzt daraufhin die beschriebenen Anforderungen um und erläutert die endgültige Implementierung des Experiments sowie dessen Auswertung.

\section{Erläuterung der Forschungsmethodik}

\textbf{Wie werden die Daten }

\subsection{Hypothesen}

\textbf{Hypothesen aus Kapitel eins als H0 und H1 Hypothesen formulieren.}

\subsection{Messverfahren}
\textbf{Welche Daten werden gemessen ?}

\textbf{Welche tests werden verwendet ?}