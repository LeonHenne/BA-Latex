\section{Forschungsmethodik}

Als Forschungsmethodik soll im Rahmen dieser Arbeit ein quantitatives Laborexperiment nach Recker \cite*[]{Recker.2021} durchgeführt werden.
Hierbei wird häufig nach dem hypothetisch-deduktiven Modell vorgegangen, in welchem Hypothesen formuliert, empirische Studien entwickelt, Daten gesammelt, Hypothesen anhand dieser evaluiert und gewonnene Erkenntnisse berichtet werden. \footcite[Vgl.][S. S.89f.]{Recker.2021}
Damit stellt das Laborexperiment eine Möglichkeit für die Untersuchung der Ursache- und Wirkungsbeziehung dar. \footcite[Vgl.][S. 106]{Recker.2021}
Dabei wird die kontrollierte Umgebung der Simulation erschaffen, deren Aufbau die unabhängige Variable des Laborexperiments darstellt.
Die Metriken, auf deren Grundlage die Leistung und die Robustheit der trainierten Policies gemessen werden, bilden im Experiment die abhängigen Variablen, zu dessen Grundlage die Forschungsfrage beantwortet wird.