\section{Forschungsmethodik}

Als Forschungsmethodik soll im Rahmen dieser Arbeit ein quantitatives Laborexperiment nach Recker \cite*[]{Recker.2021} durchgeführt werden.
Hierbei wird häufig nach dem hypothetisch-deduktiven Modell vorgegangen, in welchem Hypothesen formuliert, empirische Studien entwickelt, Daten gesammelt, Hypothesen anhand dieser evaluiert und gewonnene Erkenntnisse berichtet werden. \footcite[Vgl.][S. S.89f.]{Recker.2021}
Eine Möglichkeit der Untersuchung der Ursache- und Wirkungsbeziehung stellt das Laborexperiment dar. \footcite[Vgl.][S. 106]{Recker.2021}
Dabei wird die kontrollierte Umgebung der Simulation erschaffen, deren Aufbau die unabhängige Variable darstellt.
Die Metriken, anhand welcher die Performance und die Robustheit der trainierten Policies gemessen werden, bilden im Experiment die abhängigen Variablen.

%Neben dem quantitativen Laborexperiment wird die Literaturrecherche nach \cite[]{10.5555/2017160.2017162} durchgeführt.