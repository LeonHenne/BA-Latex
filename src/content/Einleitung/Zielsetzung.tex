\section{Zielsetzung}

Untersuchung der erreichten Generalisierung von erlernten Policies durch Hinzufügen von Agenten als regelbasierten und als mittels RL angelernten Gegenspieler in der Simulationsumgebung. Vergleich mit bekannten Methoden der Domain Adaptation oder Domain Randomization, wie dem Verändern von Dynamikparametern der Umgebung.

Dazu 

Entwicklung einer beispielhaften Simulationsumgebung zweier sich konkurrierender Spieler in Form von Flugobjekten, welche sich spielerisch gegenseitig bekämpfen. Trainieren einer Policy gegen einen Regelbasierten Gegenspieler. 
Training einer weiteren Policy zum Einsatz als Gegenspieler. 
Training einer Policy im Konkurrenzverhalten gegen den zuvor trainierten Gegenspieler.  
Vergleich der Policies aus dem Training mit regelbasiertem und RL gesteuerten Gegenspieler in einer Dynamik veränderten Umgebung.
Testen von Performance-Metriken, wie der durchschnittlich erzielten Belohnung und Stabilitäts-Metriken, wie der Belohnungsvarianz oder der Anzahl an Abstürzen.
