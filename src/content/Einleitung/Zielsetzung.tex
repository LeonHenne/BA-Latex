\section{Zielsetzung}

Daher soll im Rahmen dieser Arbeit untersucht werden, ob die Integrierung eines Gegenspielers in einer Simulation die Umgebung so beeinflussen kann, dass die erlernten Verhaltensmodelle, welche im Kontext von RL oftmals als Policies referenziert werden, robuster unter den veränderten dynamischen Bedingungen agieren. %von dem Einsatz weiteren Methoden, wie der Domain Adaptation oder 

Dazu soll eine kompetitive Simulationsumgebung entwickelt werden, in welcher sich zwei konkurrierender Spieler in Form von Flugobjekten spielerisch gegenseitig bekämpfen.
Diese ermöglicht das Training von Policies mit regelbasiertem und bereits durch RL optimierten Gegenspieler.
Unter dessen Verwendung können daraufhin Policies gegen den regelbasierten- und RL-Gegenspieler angelernt werden. 
Anschließend wird die erreichte Generalisierung mit bereits bekannten Methoden zur Erhöhung der Robustheit von Policies wie Domain Randomization verglichen.
Dazu werden Performance-Metriken, wie die durchschnittlich erzielte Belohnung und Stabilitäts-Metriken, wie die Varianz der Belohnungen oder die Anzahl an Abstürzen betrachtet. 

%Anschließend sind die Policies in einer mittels Domain Randomization veränderten Simulationsumgebung auf ihre Generalisierung zu testen.

