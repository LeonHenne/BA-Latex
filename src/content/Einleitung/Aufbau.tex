\section{Aufbau der Arbeit}

Insgesamt gliedert sich die Arbeit nach einem Schema von \cite[]{Holzweiig.2022}
Die Arbeit beginnt mit einem einleitenden Kapitel, in welchem Motivation, Problemstellung, Zielsetzung und Forschungsmethodik erläutert sind.
Anschließend wird im zweiten Kapitel der aktuelle Stand der Forschung zu den relevanten Konzepten der Problemstellung wiedergegeben.
Im dritten Kapitel wird die Forschungsmethodik dargestellt, indem die Simulationsumgebung als Messinstrument entwickelt wird sowie verschiedene Messszenarien erläutert und entsprechende Daten gesammelt werden. 
Daraufhin sind im folgenden vierten Kapitel die Messdaten auszuwerten und aufgestellte Hypothesen zu überprüfen.
Im Zuge dessen kann ebenso die Forschungsfrage anhand der Annahme oder Ablehnung der Hypothesen beantwortet werden.
Abschließend wird im letzten Kapitel ein Fazit zu den erzielten Forschungsergebnissen dargelegt und ein Ausblick auf weitere Forschung gegeben.