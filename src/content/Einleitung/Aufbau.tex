\section{Aufbau der Arbeit}

% Insgesamt gliedert sich die Arbeit nach dem Schema von \cite[]{Holzweiig.2022}.
Die Arbeit beginnt mit einem einleitenden Kapitel, in welchem Motivation, Problemstellung, Zielsetzung und Forschungsmethodik erläutert sind.
Anschließend wird im zweiten Kapitel der aktuelle Stand der Forschung zu den relevanten Konzepten beschrieben.
Im dritten Kapitel wird die Forschungsmethodik dargestellt, indem die Simulationsumgebung als Messinstrument entwickelt wird, verschiedene Messszenarien erläutert und entsprechende Daten gesammelt werden. 
Daraufhin erfolgt im vierten Kapitel die Auswertung der Messdaten sowie die Beantwortung der Forschungsfrage durch Annahme oder Ablehnung der eigens aufgestellten Hypothesen.
Im Zuge dessen kann ebenso die Forschungsfrage anhand der Annahme oder Ablehnung der Hypothesen beantwortet werden.
Abschließend wird im letzten Kapitel ein Fazit gezogen, die Forschungsmethodik kritisch reflektiert und ein Ausblick auf nachfolgende Forschungsansätze gegeben.