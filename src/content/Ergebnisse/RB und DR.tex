\section{Vergleich der Robustheit der Strategien aus dem Training mit regelbasierten Gegenspieler und Domain Randomization}

\begin{table}[htb]
    \begin{tabular}{llllll}
    Metrik & Gleichheit P-Wert & Verschlechterung P-Wert & Ungleichheit & Verbesserung \\
    sum reward & 0.48390799726891964 & 0.7588082897727775 & False & False \\
    \begin{tabular}[c]{@{}l@{}}reward mean \\ difference\end{tabular} & 2.2302103039562726e-05 & 0.9999889697144758 & True & False \\
    \begin{tabular}[c]{@{}l@{}}count of \\ failures \end{tabular} & 0.005412977104604817 & 0.0027064885523024086 & True & True
    \end{tabular}
    \caption{Ergebnisse der statistischen Tests aus dem Leistungsvergleich der Modelle aus den Trainingsszenarien 1 und 4}
    \label{tab:my-comparison-2}
\end{table}

Tabelle 8 weist die Ergebnisse der statistischen Signifikanztests zum Vergleich der Strategieleistungen aus dem Training mit regelbasiertem Gegenspieler und DR aus.
Dabei werden in gleicher Struktur zur Tabelle 7 die P-Werte zur Gleichheit und Verschlechterung der Verteilung sowie die Annahme der H1 Hypothesen anhand jeder Metrik dargestellt.
Forschungshypothesen und Testhypothesen sind entsprechend dem Vergleich der Polices aus dem Training mit DR und ohne DR anzupassen.

\subsection{Betrachtung der kumulierten Belohnung}

\textbf{Hypothese 1:}
\textit{Die im Testszenario erzielte kumulierte Belohnung ist unter Verwendung der Policy aus dem Training mit DR signifikant und zuverlässig höher als die Policy aus dem Training ohne Randomisierung der Umgebung.}

% Untersuchungsgegenstand
Für die Ermittlung des Effekts der DR wird die Forschungshypothese 1 für den Vergleich der Strategieleistungen aus dem Training mit und ohne DR adaptiert.
Die Hypothese stellt damit die Behauptung auf, dass die kumulierte Belohnung signifikant höher ist, sofern ein Training mit DR vorliegt.
Dessen Auswertung erfolgt über eine ähnliche Adaptierung der H0 und H1 Hypothesen der Tests zur Gleichheit und Verschlechterung der Ergebnissdaten.
Um die adaptierte Hypothese zu bestätigen gilt es eine Ungleichheit im ersten Signifikanztest und eine Verbesserung im zweiten Test festzustellen.

% Evaluation
Werden zur Auswertung die P-Werte aus der Tabelle 8 betrachtet liegt unter anderem aus dem Test der Gleichheit ein P-Wert von in etwa $0.4839$ vor.
Da dieser Wert über dem Signifikanzniveau liegt, ist die entsprechende Null Hypothese anzunehmen.
Daraus ist festzustellen, dass die H1 Hypothese verworfen wird und folglich von einer gleichen Verteilung der Belohnungen in Abhängigkeit des Trainingsszenarios auszugehen ist.
% Wird der P-Wert des zweiten Signifikanztests herangezogen, so wird erkenntlich, dass eine Verschlechterung der Belohnungsverteilung erzielt worden ist mittels der Strategie, welche unter DR optimiert wurde.
% Dies geht daraus hervor, dass der P-Wert mit in etwa $0.9533$ größer als das Signifikanzniveau ist.
% Im Test zur Verschlechterung der Metrik wird dadurch die H0 Hypothese bestätigt und die H1 Hypothese dem entsprechend abgelehnt.

% Bezug zur Robustheit
Für das beschriebene Testszenario konnte der Einsatz von DR im Trainingsszenario demnach keine Verbesserung der Belohnungsverteilung, hervorgerufen werden.
Die im Rahmen dieser Arbeit verwendete Form der Domänen Randomisierung zeigte somit auch keinen Effekt auf die Robustheit der Strategien gegenüber des unbekannten Testszenarios, wird lediglich die kumulierte Belohnung je Episode betrachtet.
% Aus den Ergebnissen der Signifikanztests geht hervor, dass mittels der in dieser Arbeit angewendeten DR im Trainingsprozess ein negativer Effekt auf die Belohnungsverteilung ausgeübt wurde.

\subsection{Betrachtung der Belohnungsstabilität}

\textbf{Hypothese 2:}
\textit{Die Varianz der im Testszenario erzielten kumulierten Belohnung ist unter Verwendung der Policy aus dem Training mit DR signifikant und zuverlässig geringer als die Policy aus dem Training ohne Randomisierung der Umgebung.}

% Untersuchungsgegenstand
Mittels der adaptierten zweiten Forschungshypothese wird die Verteilung der Abweichungen vom Mittelwert der Belohnungen über alle 100 Testszenarien untersucht.
Dazu wird im ersten Test die Gleichheit der Verteilungen, aus der Verwendung der beiden Strategien mit und ohne DR, als H0 Hypothese vorausgesetzt.
Im zweiten Test wird die Verringerung der Verteilung aus der Verwendung der Strategie mit DR angenommen.

% Evaluation
Tabelle 8 zeigt zum ersten Test der Gleichheit einen P-Wert von in etwa $2.2302e-05$.
Daraus resultierend können nur zu einer sehr geringen Wahrscheinlichkeit per Zufall die Daten aus einer selben unterliegenden Verteilung gezogen worden sein.
Eine gleiche unterliegende Verteilung wird demnach ausgeschlossen und die H0 Hypothese abgelehnt sowie die H1 Hypothese angenommen.
Bezogen auf den zweiten Test liegt ein P-Wert von in etwa $1.0$ vor.
Folgend kann abgeleitet werden, dass die H0 Hypothese einer stärkeren Verteilung der Abweichungen anzunehmen ist.
Entsprechend wird die H1 Hypothese sowie die Hypothese 2 abgelehnt, da die angewendete Form der DR zu einen negativen Effekt auf die Varianz der Belohnungen führt.

% Bezug zur Robustheit
Für die Messung der Robustheit ist somit festzuhalten, dass DR im Kontext dieser Arbeit als Teil des Trainingsszenarios, keine Verbesserung der Robustheit im Sinne einer stabileren Belohnung erzielte.

\subsection{Betrachtung der Anzahl der Misserfolge}

\textbf{Hypothese 3:}
\textit{Die im Testszenario erreichte Anzahl von Misserfolgen ist unter Verwendung der Policy aus dem Training mit DR signifikant und zuverlässig geringer als die Policy aus dem Training ohne Randomisierung der Umgebung.}

% Untersuchungsgegenstand
Mit der dritten adaptierten Forschungshypothese soll zuletzt der Effekt der Domänen Randomisierung auf die Anzahl der Misserfolge untersucht werden.
Durch den Test der Gleichheit wird die Existenz eines Effekts überprüft.
Sofern dieser Test einen vorhanden Effekt bestätigt, wird dessen Auswirkung mittels des zweiten Tests zur Verschlechterung evaluiert.
Damit die dritte Forschungshypothese angenommen wird, gilt es, eine verringernden Effekt auf die Häufigkeit von Misserfolgen festzustellen.
Dazu sind die H0 Hypothesen beider Tests abzulehnen.

% Evaluation
Anhand Tabelle 8 lassen sich die Ergebnissdaten der Signifikanztests ablesen.
Die Tabelle zeigt einen P-Wert von in etwa $0.0054$ zum Test der Gleichheit beider Verteilungen der Misserfolge auf.
Aus der erfüllten Bedingung das dieser Wert geringer als das Signifikanzniveau ist, kann die H0 Hypothese abgelehnt werden.
Folglich wird für den weiteren Testverlauf angenommen, dass die Verteilung der Misserfolge durch die Auswahl des Trainingsszenarios beeinflusst werden konnte.
Wird anschließend der P-Wert zur Evaluation des entstandenen Effekts von in etwa $0.0027$ betrachtet, können folgende Schlussfolgerungen gezogen werden.
Die H0 Hypothese einer höheren Häufigkeit von Misserfolgen kann abgelehnt werden, da der P-Wert im Ablehnungsbereich unterhalb des Signifikanzniveaus liegt.
Im Gegensatz dazu kann die H1 Hypothese und eine Verbesserung der Häufigkeit von Misserfolg angenommen werden.

% Bezug zur Robustheit
Aus den Annahmen und Ablehnungen der Testhypothesen ergibt sich, dass durch die Anwendung der Domänen Randomisierung eine Verbesserung der Häufigkeit von Misserfolgen erzielen lies.
Wird nur diese Metrik betrachtet kann entsprechend ausgesagt werden, dass die Strategie durch DR robuster auf unbekannte Umgebungen reagiert, da die Testepisoden häufiger erfolgreich abgeschlossen wurden.