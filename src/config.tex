% Typ der Arbeit (für Deckblatt und ehrenwörtliche Erklärung) - bitte Zutreffendes auswählen
% 1. Projektarbeit
% 2. Projektarbeit
% Seminararbeit
% Bachelorarbeit
\newcommand{\bootstrapPaperType}{Bachelorarbeit}

% Thema der Arbeit (für ehrenwörtliche Erklärung, ohne Umbrüche)
\newcommand{\bootstrapPaperTitle}{Experiment zur Verbesserung der Robustheit von Reinforcement Learning Modellen in Drohnensimulationen anhand trainiertem Gegenspieler}

% Ggf. Untertitel der Arbeit (falls dies nicht benötigt wird, das zweite Klammerpaar leer lassen, nicht das Kommando löschen!)
\newcommand{\bootstrapPaperSubtitle}{}

% Vorname, Name der Autorin/des Autors (für Titelseite und Metadaten)
\newcommand{\bootstrapAuthor}{Leon Henne}

% Abgabedatum (\today wird durch das Datum zum Zeitpunkt der Kompilation ersetzt/Datum kann auch manuell in die Klammern geschrieben werden)
% Beispiel: 10. Mai 2021
\newcommand{\bootstrapDueDate}{08. Mai 2023}

% Fakultät der Arbeit festlegen
\newcommand{\bootstrapFaculty}{Wirtschaft}

% Studiengang festlegen.
\newcommand{\bootstrapMajor}{Wirtschaftsinformatik}

% Kurs festlegen
\newcommand{\bootstrapCourseID}{WWI2020F} % Kursnummer (Beispiel: WWI2018F)

% Informationen bezüglich des dualen Partners (Arbeitsgeber/Firma)
\newcommand{\bootstrapCompanyAdvisorGender}{f} % m/f/x für die verschiedenen Geschlechter
\newcommand{\bootstrapCompanyName}{IBM Deutschland GmbH} % Name der Firma
\newcommand{\bootstrapCompanyAdvisorDetails}{Sophie Lang} % Name und Titel der betreuenden Person in der Firma.
\newcommand{\bootstrapCompanyAdvisorPosition}{Senior Data Scientist} % Funktion der betreuenden Person in der Firma (Job Titel).

% Informationen bzgl. der wissenschaftliche Betreuung
\newcommand{\bootstrapUniversityAdvisorDetails}{Prof. Dr. Kai Holzweißig} % Name und Titel der Person der wiss. Betreuung.
\newcommand{\bootstrapUniversityAdvisorPosition}{Studiendekan Wirtschaftsinformatik} % Funktion der Person der wiss. Betreuung.
